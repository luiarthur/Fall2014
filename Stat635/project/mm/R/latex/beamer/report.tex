\documentclass{beamer}
%\usepackage{Sweave}                                                       
\usepackage{amssymb,bm,mathtools,amsmath}                                                      
\usepackage{graphicx,caption,float}
\usepackage[UKenglish]{isodate} % for: \today                             
\cleanlookdateon                % for: \today                             
                                                                          
\def\wl{\par \vspace{\baselineskip}\noindent}                             
\def\beginmyfig{\begin{figure}[ht]\begin{center}}                          
\def\endmyfig{\end{center}\end{figure}}                                   

\def\prodl#1#2#3{\prod\limits_{#1=#2}^{#3}}                               
\def\suml#1#2#3{\sum\limits_{#1=#2}^{#3}}                                 
\def\ds{\displaystyle}                                                    
\def\tbf#1{\textbf{#1}}
\def\inv{^{\raisebox{.2ex}{$\scriptscriptstyle-1$}}}
\def\pm{^{\raisebox{.2ex}{$\scriptscriptstyle\prime$}}}

% Beamer Stuff:
\usepackage{beamerthemeBoadilla} % Determines the Theme
\usecolortheme{beaver}           % Determines the Color
%\usefonttheme{serif}

\beamertemplatenavigationsymbolsempty % To get rid of navigation bar

% my title:                                                               
\title[IBP in Mixed Models]{Using the Indian Buffet Process to Estimate the Design Matrix Z in
                            Random Intercept Mixed Models}
%\logo{\includegraphics[width=1cm,height=1cm,keepspectration]{/data/arthurll/Pictures/logo.png}}
\author[Arthur Lui]{Arthur Lui}
\institute[Brigham Young University]{
  Department of Statistics\\
  Brigham Young University
}
%\setkeys{Gin}{width=0.5\textwidth}                                       
\begin{document}                                                          
%%%%%%%%%%%%%%%%%%%%%%%%%%%%%%%%%%%%%%%%%%%%%%%%%%%%%%%%%%%%%%%%%%%%%%%%%%%

\frame{\titlepage}

\frame{
  \frametitle{Observations}
  \beginmyfig
    \caption{Simulated Observations}
    \includegraphics[scale=.4]{../images/scatter.pdf}
  \endmyfig
}    

\frame{
  \frametitle{Random Intercept Linear Mixed Models}
  \beginmyfig
    \caption{True Groups}
    \includegraphics[scale=.4]{../images/clus.pdf}
  \endmyfig
} 



%\frame{
%  \frametitle{Mixed Model}
%  \[\bm{ y = X\beta + Z \gamma + \epsilon }\]
%  \beginmyfig
%    \vspace{-10mm}
%    \includegraphics[scale=.4]{../images/clus.pdf}
%  \endmyfig
%}

\frame{
  \frametitle{Random Intercept Linear Mixed Models}
  \begin{columns}
    \begin{column}{.4\textwidth}
      \beginmyfig
        \includegraphics[scale=.3]{../images/clus.pdf}
      \endmyfig
    \end{column}
    \begin{column}{.6\textwidth}
      \begin{itemize}
        \setlength\itemsep{1em}
        \item $\bm{y = X\beta + Z \gamma + \epsilon}$
        \begin{itemize}
          \setlength\itemsep{1em}
          \pause
          %\item[]
          \item $\bm{V=Z\hat GZ\pm+\hat R}$
          \item $\bm{\hat\beta=(X\pm V\inv X)\inv X\pm V\inv y}$
          \item $\bm{\hat\gamma=\hat G Z\pm V\inv (y-X\hat\beta)}$
        \end{itemize}
        \item []
      \end{itemize}
    \end{column}
  \end{columns}
}

\frame{
  \frametitle{Random Intercept Linear Mixed Models}
  \begin{columns}
    \begin{column}{.4\textwidth}
      \beginmyfig
        \includegraphics[scale=.3]{../images/truemm.pdf}
      \endmyfig
    \end{column}
    \begin{column}{.6\textwidth}
      \begin{itemize}
        \setlength\itemsep{1em}
        \item $\bm{y = X\beta + Z \gamma + \epsilon}$
        \begin{itemize}
          \setlength\itemsep{1em}
          %\item[]
          \item $\bm{V=Z\hat GZ\pm+\hat R}$
          \item $\bm{\hat\beta=(X\pm V\inv X)\inv X\pm V\inv y}$
          \item $\bm{\hat\gamma=\hat G Z\pm V\inv (y-X\hat\beta)}$
        \end{itemize}
        \pause
        \item But what if we don't know $\bm Z$?
      \end{itemize}
    \end{column}
  \end{columns}
}

\frame{
  \frametitle{Random Intercept Linear Mixed Models}
  \beginmyfig
    \caption{True Groups}
    \includegraphics[scale=.4]{../images/clus.pdf}
  \endmyfig
} 
\frame{
  \frametitle{Random Intercept Linear Mixed Models}
  \beginmyfig
    \caption{Groups obtained by kmeans}
    \includegraphics[scale=.4]{../images/kmean.pdf}
  \endmyfig
} 

\frame{
  \frametitle{Indian Buffet Process (IBP)}
  $\underset{N\times\infty}{\bm Z} \sim \text{IBP}_N(\alpha)$,
  where $z_{ik} \in \left\{0,1\right\}$
  \pause
  \wl
  \beginmyfig
    \includegraphics[scale=.3]{../images/ibpExample.pdf}
  \endmyfig
}

\frame{
  \frametitle{Model}
  \begin{align*}
    \text{Likelihood:}\\
    \bm{y|X,\beta,Z,\gamma},\sigma_r^2 &\sim 
      N(\bm{X\beta+Z\gamma},\sigma_r^2 \bm I_N)\\
    \text{Priors:}\\
    \bm\beta &\sim N(0,s_\beta^2\bm I_2)\\
    \bm Z|\alpha &\sim \text{IBP($\alpha$)}\\
    \alpha &\sim \text{Gamma}(a_\alpha,b_\alpha)\\
    \bm\gamma &\sim N(0,s_\gamma^2\bm I_{K?})\\
    \sigma_r^2 &\sim \text{Gamma}(a_\sigma,b_\sigma)\\
    \\
    \text{Marginalized Likelihood:}\\
    \bm{[y|Z,X},\sigma_r^2] &= \int_{\beta,\gamma} \bm{[y|Z,X,\beta,\gamma][\beta][\gamma]} 
                               d\bm\beta d\bm\gamma\\
  \end{align*}
}

\frame{
  \frametitle{Results}
  \beginmyfig
    \includegraphics[scale=.4]{../images/agpost.pdf}
  \endmyfig  
}

\frame{
  \frametitle{Results}
  \beginmyfig
    \includegraphics[scale=.4]{../images/EZpre.pdf}
    \vspace{-5mm}
    \caption{Estimated Z Matrix}
  \endmyfig  
}

\frame{
  \frametitle{Results}
  \beginmyfig
    \includegraphics[scale=.4]{../images/EZ.pdf}
    \vspace{-5mm}
    \caption{Reparameterized Estimated Z Matrix}
  \endmyfig  
}

\frame{
  \frametitle{Results}
  \beginmyfig
    \includegraphics[scale=.4]{../images/KZ.pdf}
    \vspace{-5mm}
    \caption{Z Matrix (K-Means)}
  \endmyfig  
}

\frame{
  \frametitle{Results}
  \begin{columns}
    \begin{column}{.6\textwidth}
      \beginmyfig
        \caption{True}
        \vspace{-5mm}
        \includegraphics[scale=.4]{../images/truemm.pdf}
      \endmyfig
    \end{column}
    \begin{column}{.4\textwidth}
      \input{../images/mb.tex}
      \input{../images/mg.tex}     
    \end{column}
  \end{columns}
}
\frame{
  \frametitle{Results}
  \begin{columns}
    \begin{column}{.6\textwidth}
      \beginmyfig
        \caption{IBP}
        \vspace{-5mm}
        \includegraphics[scale=.4]{../images/ibpmm.pdf}
      \endmyfig
    \end{column}
    \begin{column}{.4\textwidth}
      \input{../images/mb.tex}
      \input{../images/mg.tex}     
    \end{column}
  \end{columns}
}
\frame{
  \frametitle{Results}
  \begin{columns}
    \begin{column}{.6\textwidth}
      \beginmyfig
        \caption{K-Means}
        \vspace{-5mm}
        \includegraphics[scale=.4]{../images/KMmm.pdf}
      \endmyfig
    \end{column}
    \begin{column}{.4\textwidth}
      \input{../images/mb.tex}
      \input{../images/mg.tex}     
    \end{column}
  \end{columns}
}

\frame{
  \frametitle{Future Work}
  \begin{itemize}
    \item Proposal mechanism for IBP
    \item Non-normal likelihood
    \item Applications
    \item Random slope?
  \end{itemize}
}

\frame{
  \frametitle{Questions?}
  \begin{center}
    \Huge Thank You!
  \end{center}
}

  \frame{
    \frametitle{Indian Buffet Process (IBP)}
    Customer $i$ taking a dish $k$ is analogous to observation $i$ possessing
    feautre $k$. For $\bm Z \sim \text{IBP}_N(\alpha)$:\\
    \wl
    \begin{enumerate}
      \setlength\itemsep{1em}
      \pause
      \item The first customer takes Poisson($\alpha$) number of dishes
      \pause
      \item For customers $i=2 \text{ to } N$,
      \begin{itemize}
        \pause
        \item For each previously sampled dish, customer $i$ takes dish $k$
              with probability $m_{-i,k} / i$
              \pause
        \item after sampling the last sampled dish, customer $i$ samples 
              Poisson($\alpha/i$) new dishes
      \end{itemize}
    \end{enumerate}
  }

  \frame{
    \frametitle{Realizations from the IBP}
    \beginmyfig
      %\includegraphics[scale=.4]{ibp2.pdf}
    \endmyfig
  }

  \frame{
    \frametitle{PMF for the IBP}
      \[
      \begin{array}{rcl}
        \underset{N \times \infty}{\bm Z} &\sim& \text{IBP}(\alpha),
           \text{ where $z_{ik} \in \{0,1\}$} \\
        \implies \text{P}(\bm Z) & = & \frac{\alpha^{K}}{\prodl{i}{1}{N} 
                                       x_i!} 
                                       exp\{-\alpha H_N\}\prodl{k}{1}{K}
                                       \frac{(N-m_k)!(m_k-1)!}{N!}\\
                                 \pause
                                 & = & \frac{\alpha^{K}exp\{-\alpha H_N\}}
                                            {\prodl{i}{1}{N}(x_i!~i^{x_i})} 
                                       \prodl{i}{2}{N}\prodl{k}{1}{y_{i}}
                                       \left(\frac{m_{-i,k}}{i}\right)^{z_{i,k}}
                                       \left(1-\frac{m_{-i,k}}{i}\right)^{1-z_{i,k}}
      \end{array}
      \]
      \pause
      where $H_N$ is the harmonic number $\suml{i}{1}{N}i^{-1}$, $K$ is the
      number of non-zero columns in $\bm Z$, $m_k$ is the $k^{th}$ column sum
      of $\bm Z$, $y_i$ is the total number of dishes sampled before the
      $i^{th}$ customer, $m_{-i,k}$ is the number of customers that sampled
      dish k before customer $i$,and $x_i$ is the number of ``new dishes"
      sampled by customer $i$.\\
  }

\end{document}
