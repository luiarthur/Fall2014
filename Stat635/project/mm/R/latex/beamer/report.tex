\documentclass{beamer}                                                    
\usepackage{amssymb}                                                      
%\usepackage{Sweave}                                                       
\usepackage{bm}                                                           
\usepackage{mathtools}                                                    
\usepackage{graphicx}
\usepackage{float}
\usepackage{amsmath} % multline
\usepackage{caption} %captionof
\usepackage[UKenglish]{isodate} % for: \today                             
\cleanlookdateon                % for: \today                             
                                                                          
\def\wl{\par \vspace{\baselineskip}\noindent}                             
\def\beginmyfig{\begin{figure}[ht]\begin{center}}                          
\def\endmyfig{\end{center}\end{figure}}                                   

\def\prodl#1#2#3{\prod\limits_{#1=#2}^{#3}}                               
\def\suml#1#2#3{\sum\limits_{#1=#2}^{#3}}                                 
\def\ds{\displaystyle}                                                    

% Beamer Stuff:
\usepackage{beamerthemeBoadilla} % Determines the Theme
\usecolortheme{beaver}         % Determines the Color
\beamertemplatenavigationsymbolsempty % To get rid of navigation bar

% my title:                                                               
\title[IBP In Mixed Models]{Using The Indian Buffet Process to Estimate the Design Matrix G in
                            Random Intercept Mixed Models}
%\logo{\includegraphics[width=1cm,height=1cm,keepspectration]{/data/arthurll/Pictures/logo.png}}
\author[Arthur Lui]{Arthur Lui}
\institute[Brigham Young University]{
  Department of Statistics\\
  Brigham Young University
}
%\setkeys{Gin}{width=0.5\textwidth}                                       
\begin{document}                                                          
%%%%%%%%%%%%%%%%%%%%%%%%%%%%%%%%%%%%%%%%%%%%%%%%%%%%%%%%%%%%%%%%%%%%%%%%%%%

\frame{\titlepage}
\section{Introduction}
  \frame{
    \frametitle{Why use the IBP?}
    \begin{itemize}
      \item[] Factor Analysis 
      \begin{itemize}
          \item[]
          \item[] \item Only $\bm y$ is observed 
          \item[]
          \item $\bm{Y = F\Lambda + \epsilon}$, where $\bm F$ and $\bm\Lambda$ 
                         are matrices with continuous values
          \item[]
          \item Latent Feature Model
          \begin{itemize}
            \item $\bm{Y = Z\Lambda + \epsilon}$, where $\bm Z$ is a binary matrix
            %\item $\bm{y = X\beta + Zf + \epsilon}$
            \item $\bm Z$ can be modeled with an Indian buffet process prior
          \end{itemize}
      \end{itemize}
