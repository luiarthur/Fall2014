\documentclass{article}
\usepackage{fullpage}
\usepackage{pgffor}
\usepackage{amssymb}
\usepackage{bm}
\usepackage{mathtools}
\usepackage{verbatim}
\usepackage{appendix}
\usepackage{graphicx}
\usepackage[UKenglish]{isodate} % for: \today
\cleanlookdateon                % for: \today

\def\wl{\par \vspace{\baselineskip}\noindent}
\def\beginmyfig{\begin{figure}[htbp]\begin{center}}
\def\endmyfig{\end{center}\end{figure}}
\def\myfig#1{\beginmyfig\includegraphics{#1}\endmyfig}
%\def\prodl{\prod\limits_{i=1}^n}
\def\suml{\sum\limits_{i=1}^n}
\def\ds{\displaystyle}

\begin{document}
% my title:
\begin{center}
  \section*{\textbf{Stat637: Multilevel (Hierarchical) Modeling: What It Can and Cannot Do}
    \footnote{https://github.com/luiarthur/Fall2014/blob/master/Stat637/mp}
  }  
  \subsection*{\textbf{Arthur Lui}}
  \subsection*{\noindent\today}
\end{center}

\section{Summary of the Paper}
Gelman reviews the multilevel (hierarchical) model using a radon-measurement
dataset from the Environmental Protection Agency (EPA). He demonstrates that
using the hierarchical model is almost always an improvement from classical
regression, and shows when it is essential, useful, or only helpful.\\

\noindent
Of interest to the EPA is the distribution of radon levels across homes in the
US. Radon is a carcinogenic gas that causes several thousand lung cancer deaths
each year. It is known that radon comes from underground and enters more easily
into homes that are built into the ground, or that have basements.
Consequently, the presence of basements in homes an important predictor for
radon levels. In addition, uranuim, a solid that exists in soil, is a parent
element that eventually decays to form radon gas. So, soil uranium measurements
are also an important predictor of radon levels. The dataset from the EPA is
contains information on more than 80,000 houses throughout the country. But
Gelman analyzes only data in Minnesota, and groups observations (to form a
hierarchy) by county, within Minnesota.\\

\noindent

\section{How the Paper Relates to Generalized Linear Models}

\section{Fitting The Model}
\myfig{images/apost.pdf}
\myfig{images/bpost.pdf}
\myfig{images/gpost.pdf}
\myfig{images/sy2post.pdf}
\myfig{images/sa2post.pdf}
\myfig{images/au.pdf}

\section{Extending Results of the Paper}

\end{document}
