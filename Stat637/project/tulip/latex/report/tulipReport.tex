\documentclass[mathserif]{article}
\usepackage{fullpage}
\usepackage{pgffor}
\usepackage{amssymb}
\usepackage{bm}
\usepackage{mathtools}
\usepackage{verbatim}
\usepackage{appendix}
\usepackage{graphicx}
\usepackage{amsmath}
\usepackage[UKenglish]{isodate} % for: \today
\cleanlookdateon                % for: \today
\newcommand\numberthis{\addtocounter{equation}{1}\tag{\theequation}}

\def\wl{\par\vspace{\baselineskip}\noindent}
\def\beginmyfig{\begin{figure}[htbp]\begin{center}}
\def\endmyfig{\end{center}\end{figure}}
\def\prodl#1#2#3{\prod\limits_{#1=#2}^{#3}}
\def\suml#1#2#3{\sum\limits_{#1=#2}^{#3}}
\def\ds{\displaystyle}
\def\tbf#1{\textbf{#1}}
\def\inv{^{\raisebox{.2ex}{$\scriptscriptstyle-1$}}}
\def\pm{^{\raisebox{.2ex}{$\scriptscriptstyle\prime$}}}
\newcommand{\m}[1]{\mathbf{\bm{#1}}} % Serif bold math

\begin{document}
% my title:
\begin{center}
  \section*{\textbf{Predicting Germination Rates of Different Tulip
                    Populations at Various Chilling Times}
    \footnote{https://github.com/luiarthur/Fall2014/tree/master/Stat637/project/tulip}
  }
  \subsection*{\textbf{Arthur Lui}}
  \subsection*{\noindent\today}
\end{center}

\section*{Introduction}
Tulips popularized in the sixteenth century in Holland. During the tulipomania,
a viceroy bulb could allegedly be exchanged for a basket of goods, some
furniture, \textit{and} some live stock. Today, nine million bulbs are produced
annually, and tulips account for 25\% of agricultural exports. Many tourists
also come to the Netherlands to see the vast tulip fields each year. Since
tulips are such a prominant part of the Dutch economy, the country is axious
about maximizing the growth of these beautiful flowers.\\

\noindent
A major factor that affects the growth of tulips is termed the ``chill time" of
the tulip seeds. Chill time is defined as the time that a (tulip) seed is
present in temperatures below 55$^\circ$F, prior to germinating. If
insufficient chill time is given to tulip seeds, they are not as likely to
germinate and flower. On the other hand, too much chill time can also adversely
affect the germination rates of the seeds.  Most tulips need 12-14 weeks of
chill time. For a given variety of tulip, a population of tulips may exhibit different
responses to chill times due to its conditioning. That is, some tulip populations
may be more robust to adverse weather effects than others. Hence, scientists are
interested in identifying tulip populations that are more resilient so as to 
breed those types of tulips more widely. Specifically, due to global warming, 
scientists want to identify the tulip populations that are able thrive most
in warmer weathers, and require a shorter chill time.\\

\noindent
In order to identify tulip populations that are most resilient to warm weather
climates, we were given a dataset from Dr. Matthew Heaton, containing
information of eleven tulip populations at a research farm in the Netherlands.
Each tulip population was treated with seven chilling times (0,2,4,6,8,10, and
12 weeks). These chilling times were artificially created by placing the tulip
seeds in chillers. For each of the eleven populations, thirty seeds were
observed for each of the seven chilling times. So, a total of 2310 = 30*7*11
observations are contained in this dataset. In the dataset, information on each
of the 2310 tulip seeds is provided. One column indicates weather the seed
germinated (Y/N). Another column indicates the population the seed belongs to
(1-11). And yet another column shows the chill time given for each seed
(0,2,4,6,8,10,12). Two other variables containing the dates the tulip seeds
were harvests were not used for this study. Figure 1 shows the germination
rates over all populations at the 7 different chill times (top left), and
the germination rates for each of the eleven populations at the 7 different
chill times.\\
\beginmyfig
    \includegraphics[scale=.35]{../../images/rawData.pdf}
    \caption{Germination rates over all populations at the 7 different
             chill times (top left), and the germination rates for each of 
             the eleven populations at the 7 different chill times.}
\endmyfig

\noindent
For this project, our goal is to identify (1) the effect of chill time for each
tulip population, (2) the optimal chill time for each population, and (3) how
the optimal chill times differ by population. To determine these items, we will
use a bayesian probit model for our data, with chilling time and population as
our predictors, and whether the seed germinated as our response. We will
introduce the model in the following section, and provide interpretation the
results of our analysis in the next section.\\

\section*{Model}
Since we are modeling (germination) rates, it would be appropriate to use a
logistic regression or a bayesian probit model. After fitting both models, I
decided to use the bayesian probit model I decided to use the bayesian probit
model due to the ease of obtaining interpretable credible intervals. In the 
following subsection, we outline the model definition.\\

\subsection*{Model Definition}
We model $y_i$ with a Bernoulli($p_i$) distribution. And place a probit link on 
$p_i$. That is the link function is the inverse cummulative distribution function
of the standard normal distribution. We set the link function to be the linear combination
of the splined covariates $\m{b(x_i)}$ in the design matrix and the coefficients $\m{\beta}$. 
Here, $\m{x_i'}$ is a vector of length 2, and is (1,$x_i$). We will define $\m{b(x_i')}$ to
be a vector of length 48. The first element is 1. The next three elements are the
resulting values of a basis expansion for a cubic polynomial. The next 44 elements
are either 0,1, or $b(x_i)$, depending on the tulip population $i$. After specifying
the likelihood, we specify a prior for $\m{\beta}$. We will put a Normal prior on 
$\m{\beta}$ centered at 0. Below, we outline the model definition in mathematical 
terms.\\
\begin{center}
  \begin{tabular}{rl}
    $y_i  \sim$& \text{Bernoulli}($p_i$)\\
    $\Phi\inv(p_i) =$&$ \m{b(x_i')\beta}$ \\
                  $=$&$ \beta_0 + \suml{j}{1}{3}\beta_j b_j(x_i) +
                        \suml{k}{1}{11} \beta_{0j}\cdot I\{\text{pop}_i=k\} + $\\
                $ $  &$ \suml{k}{1}{11} \suml{j}{1}{3}\beta_{1j}\cdot I\{\text{pop}_i=k\} b_j(x_i)$\\
                 &\\
    \text{Let $\m{W=b(X)}$}, & \\
    $\m{\beta}\sim$&$\text{Normal}\left(\m{0},s_b^2\m{(W\pm W)\inv}\right)$
  \end{tabular}
\end{center}

\subsection*{Model Diagnostics}
\subsection*{Model Comparison}
\subsection*{Model Interpretation / Variable Selection}
\subsection*{Parameter Interpretation}

\section*{Results}
\beginmyfig
  \includegraphics[scale=.35]{../../images/chilleffect.pdf}
  \vspace{-2mm}
  \caption{Predicted germination rates of each population of tulips by
           chilling times (weeks)}
\endmyfig
\section*{Conclusion}

\end{document}
