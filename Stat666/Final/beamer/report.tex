\documentclass{beamer}                                                    
\usepackage{fullpage}                                                     
\usepackage{pgffor}                                                       
\usepackage{amssymb}                                                      
\usepackage{Sweave}                                                       
\usepackage{bm}                                                           
\usepackage{mathtools}                                                    
\usepackage{verbatim}                                                     
\usepackage{appendix}                                                     
\usepackage{graphicx}
\usepackage{float}
\usepackage[UKenglish]{isodate} % for: \today                             
\cleanlookdateon                % for: \today                             
                                                                          
\def\wl{\par \vspace{\baselineskip}\noindent}                             
%\def\beginmyfig{\begin{figure}[!Htbp]\begin{center}}                     
\def\beginmyfig{\begin{figure}[H]\begin{center}}                          
\def\endmyfig{\end{center}\end{figure}}                                   
\def\prodl#1#2#3{\prod\limits_{#1=#2}^{#3}}                               
\def\suml#1#2#3{\sum\limits_{#1=#2}^{#3}}                                 
\def\ds{\displaystyle}                                                    
                                                                          
% Beamer Stuff:
\usepackage{beamerthemeHannover} % Determines the Theme
\usecolortheme{seahorse}         % Determines the Color

% my title:                                                               
\title{Stat666 Final Project - The Indian Buffet Process}
\author[Arthur Lui]{Arthur Lui}
\institute[Brigham Young University]{
  Department of Statistics\\
  Brigham Young University
}
%\setkeys{Gin}{width=0.5\textwidth}                                       
\begin{document}                                                          
%%%%%%%%%%%%%%%%%%%%%%%%%%%%%%%%%%%%%%%%%%%%%%%%%%%%%%%%%%%%%%%%%%%%%%%%%%%



  \frame{\titlepage}

  \section{Introduction} % Sorah
      \frame{
        \frametitle{Introduction}
         Hello
      }
      
  \section{Data}    
      \frame{
      \frametitle{Tulip Germination Experiment}
        \begin{itemize}
          \item Goal: Understand the effect of chilling time on germination of 
                tulip bulbs.
          \wl
          \item 
            Data: 
            \begin{itemize}
              \item 12 populations each with 210 tulips (2005-2009)
              \item Each population randomly and evenly split into 7 groups 
                    and assigned to one of 7 chilling times 
                    (0, 2, 4, \ldots, 12 weeks).
              \item Response Variable: Indicator (bulb germinated or not).
              \item Population 12 did not germinate at all, so it was removed from
                    the analysis.
            \end{itemize}
        \end{itemize}    
      }
% Between 6 & 8 pages
%\section{Introduction} % < 1 page
%One key problem in recovering the latent structure responsible for generating
%observed data is determining the number of latent features. The Indian Buffet
%process (IBP) provides a flexible distribution for sparse binary matrices with
%infinite dimensions (i.e. finite number of rows, and infinite number of columns).
%When used as a prior distribution in a latent feature model, the IBP can
%learn the number of latent features generating the observations because it can
%draw binary matrices which have a potentially infinite number of columns. I will
%use the IBP as a prior distribution in a Gaussian latent feature model to
%recover the latent structures generating the observations.\\
%
%\section{Model} % 3-4 pages
%The IBP is a distribution for sparse binary matrices with finite number of rows
%and potentially infinite number of columns. The process of generating a
%realization from the IBP can be described by an analogy involving Indian buffet
%restaurants.\\
%
%\noindent
%Let $Z$ be an $N \times \infty$ binary matrix. Each row in $Z$ represents a
%customer which enters an Indian buffet and each column represents a dish in the
%buffet. Customers enter the restaurant one after another. The first customer
%samples an $r=Poisson(\alpha)$ number of dishes, where $\alpha > 0$ is a mass
%parameter which influences the final number of sampled dishes. This is
%indicated in by setting the first r columns of the first row in $Z$ to be $1$.
%The other values in the row are set to $0$. Each subsequent customer samples
%each previously sampled dish with probability proportional to its popularity.
%That is, the next customer samples dish $k$ with probability $\frac{m_k}{i}$,
%where $m_k$ is the number of customers that sampled dish $k$, and $i$ is the
%current customer number (or row number in $Z$). Each customer also samples an
%additional $Poisson(\alpha/i)$ number of new dishes. Once all the $N$ customers
%have gone through this process, the resulting $Z$ matrix will be a draw from
%the Indian buffet process with mass parameter $\alpha$. In other words, $Z \sim
%IBP(\alpha)$. Note that $\alpha \propto K_+$, where $K_+$ is the final number of
%sampled dishes (occupied columns). Figure1 shows a draw from an IBP(10) with 50
%rows. The white squares are 1, indicating that a dish was taken; black squares
%are 0, indicating that a dish was not taken. \\
%\beginmyfig
%\caption{IBP($N=50$, $\alpha=10$)}
%%\vspace{-5mm}
%<<fig=T,echo=F,height=3.7>>=
%  opts <- par()
%    #par(bg="blue",mar=c(1,1,1,1))
%    par(mar=c(0,0,0,0))
%    a.image(rIBP(50,10))
%  par(opts)
%@
%\endmyfig
%
%\noindent
%The probability of any particular matrix produced from this process is
%\begin{equation}
%  P(\bm{Z}) = \frac{\alpha^{K_+}}{\prodl{i}{1}{N} {K_1}^{(i)}!} 
%              exp\{-\alpha H_N\}\prodl{k}{1}{K_+}
%              \frac{(N-m_k)!(m_k-1)!}{N!},
%\end{equation}
%where $H_N$ is the harmonic number, $\suml{i}{1}{N}\ds\frac{1}{i}$, $K_+$ is
%the number of non-zero columns in $\bm Z$, $m_k$ is the $k^{th}$ column sum of
%$\bm Z$, and $K_1^{(i)}$ is the ``number of new dishes" sampled by customer $i$.\\
%
%\noindent
%One way to get a draw from the IBP($\alpha$) is to simulate the process according to 
%the description above. Another way is to implement a Gibbs sampler as follows:
%
%\begin{enumerate}
%  \item Start with an arbitrary binary matrix of $N$ rows
%  \item For each row, $i$,
%  \begin{enumerate}
%    \item For each column, $k$,
%    \item if $m_{-i,k}$ = $0$, delete column $k$. Otherwise,
%    \item set $z_{ik}$ to $0$
%    \item set $z_{ik}$ to $1$ with probability $P(z_{ik}=1|\bm{z_{-i,k}}) = \frac{m_{-i,k}}{i}$
%    \item at the end of row $i$, add Poisson($\ds\frac{\alpha}{N}$) columns of $1$'s
%  \end{enumerate}
%  \item iterate step 2 a large number of times
%\end{enumerate}
%We can likewise incorporate this Gibbs sampler to sample from the posterior distribution P($\bm{Z|X}$)
%where $\bm Z \sim IBP(\alpha)$ by sampling from the complete conditional
%\begin{equation}
%  P(z_{ik}=1|\bm{Z_{-(ik)},X})  \propto p(\bm{X|Z}) P(z_{ik}=1|\bm{Z_{-(ik)}}).
%\end{equation}\\
%
%
%\noindent
%Note that the conjugate prior for $\alpha$ is a Gamma distribution.
%\[
%  \begin{array}{rcl}
%    \bm Z|\alpha & \sim & IBP(\alpha)\\
%          \alpha & \sim & Gamma(a,b), \text{where $b$ is the scale parameter}\\
%    & & \\
%    p(\alpha|\bm Z) & \propto & p(\bm Z|\alpha) p(\alpha)\\
%    p(\alpha|\bm Z) & \propto & \alpha^{K_+} e^{-\alpha H_N}  
%                                \alpha^{a-1} e^{-\alpha/b}\\
%    p(\alpha|\bm Z) & \propto & \alpha^{a+K_+-1} e^{-\alpha(1/b+H_N)}\\
%  \end{array}
%\]
%\begin{equation}
%  \alpha|\bm Z  \sim  Gamma(a+K_+, (1/b+H_N)^{-1})
%\end{equation}
%
%\noindent
%%\section{Example (Data Analysis)} % 2-3 pages
%\section{Example: Linear-Gaussian Latent Feature Model with Binary Features}
%Suppose, we observe an $N \times D$ matrix $\bm X$, and we believe
%\[
%  \bm X = \bm{ZA} + \bm E,
%\]
%where $\bm Z|\alpha \sim IBP(\alpha)$, 
%$\bm A \sim MVN(\bm 0,{\sigma_A}^2\textbf{I})$, 
%$\bm E \sim MVN(\bm 0,{\sigma_X}^2\textbf{I})$.\\ 
%$\alpha ~ Gamma(a,b)$,
%%$\sigma_A=1$, and $\sigma_X=.5$.\\
%
%\noindent
%It can be shown that
%\begin{equation}  
%  p(\bm{X|Z}) = \ds\frac{1}{(2\pi)^{ND/2}\sigma_X^{(N-K)D}\sigma_A^{KD}
%                            |\bm Z^T \bm Z+({\frac{\sigma_X}{\sigma_A}})^2
%                            \textbf{I}|^{D/2}}
%                \exp\{-\frac{1}{2\sigma_X^2}tr(\bm X^T
%                (\textbf{I}-\bm Z(\bm Z^T\bm Z+({\frac{\sigma_X}{\sigma_A}})^2
%                            \textbf{I})^{-1}\bm Z))
%                \bm X\}            
%\end{equation}
%Now, we can use equation (2) to implement a Gibbs sampler to draw from the 
%posterior $\bm{Z|X},\alpha$.
%
%\subsection{Data \& Results}
%Each of ten Stat666 students created a $6 \times 6$ binary image. To the
%image, Gaussian noise (mean=0,variance=.25) was added to each cell of the
%binary image. Gaussian noise was added to the same image to generate 10 $6
%\times 6$ images. These images were turned into $1 \times 36$ row vectors. All
%these vectorized images were stacked together to form one large $100 \times 36$
%matrix.  (Note that I did not know the design of these images before hand. The
%images could be letters, numbers, interesting patterns, pokemon faces, etc.)
%Figure2 displays the data from the ten Stat666 students.\\
%
%\beginmyfig
%  \vspace{-5mm}
%  \caption{Data From Ten Stat666 Students}
%  \vspace{-2mm}
%  \includegraphics{../code/draw.post.out/Y.pdf}
%  \vspace{-15mm}
%\endmyfig
%
%\noindent
%A Gibbs sampler was implemented to retrieve posterior distributions for $\bm Z$,
%$\bm A$, and $\alpha$. $\alpha$ was initially set to $1$, and the posterior for
%$\alpha$ was obtained by equation (3) with prior distribution $Gamma(3,2)$.
%The parameters were chosen such that $\alpha$ was centered at 6 and had a
%variance of 12, as I believed I may have many latent features, but my
%uncertainty was high.  Equation(2) was used to retrieve the posterior
%distribution for $\bm Z$ (a collection of binary matrices). After 5000
%iterations, the trace plot for the number of columns in the binary matrices
%drawn were plotted (See Figure 3). Diagnostics for a cell by cell trace plot
%could also be plotted, but may be too difficult to analyze as the dimensions of
%the matrices are changing, and the matrices are large. The execution time was
%about 4 hours.\\
%\beginmyfig
%  \caption{}
%  \includegraphics{../code/draw.post.out/traceplot.pdf}
%  \vspace{-15mm}
%\endmyfig
%\beginmyfig
%  \caption{}
%  \includegraphics{../code/draw.post.out/postAlpha.pdf}
%\endmyfig
%\noindent
%The number of columns in $\bm Z$ appears to have converged to 9. This means
%that the number of latent features discovered appears to be 9. This is
%reasonable as the image $\bm X$ is comprised of 10 students' images. A burn-in
%of the first 1000 draws were removed. Then, the 4000 $\bm Z$ matrices were
%superimposed, summed element by element, and divided by 4000. Cells that had
%values $> .9$ were set to 1; and 0 otherwise. This is not necessary, but this
%removes the columns that were not likely to exist. To elaborate, from the trace
%plot (Figure3), we see that after burn-in, there is one instance where the
%number of columns in $\bm Z$ was $10$.  One instance out of 4000 is not
%significantly large enough to say that \textit{that} latent feature is 
%generating the observed data. So, the tenth column was removed. The resulting
%matrix, I will call the posterior mean for $\bm Z$.  The trace plot for
%$\alpha$ (Figure 4) shows that $\alpha$ appears to have converged, with mean
%$=2.08$ and variance $=.359$.  Figure 5 shows the posterior mean for $\bm Z$.
%We can interpret the matrix in the following way. All the observations are 
%being generated by the first column (feature). The $11^{th} through 20^{th}$
%observations in $\bm X$ are being generated by the second column, etc.
%The posterior mean for $\bm A$ calculated as $E[\bm{A|X,Z}] = (\bm Z^T\bm Z +
%\frac{\sigma_X^2}{\sigma_A^2} \textbf{I})^{-1}\bm Z^T \bm X$, and is shown
%in Figure 6 and Figure 7. Figure 6 shows the matrix in a $10 \times 36$ form;
%Figure 7 shows the matrix in $6 \times 6$ form. That is, each row in $\bm A$
%were back-transformed into its matrix form.\\
%
%\beginmyfig
%  \caption{}
%  \includegraphics{../code/draw.post.out/postZ.pdf}
%  \vspace{-10mm}
%\endmyfig
%
%\beginmyfig
%  \caption{}
%  \includegraphics{../code/draw.post.out/postA.pdf}
%\endmyfig
%\beginmyfig
%  \caption{The latent features turned back into $6 \times 6$ images.}
%  %\includegraphics[height=1\textwidth]{../code/draw.post.out/postA66.pdf}
%  \includegraphics{../code/draw.post.out/postA66.pdf}
%\endmyfig
%
%\noindent
%Now, we can predict the latent features generating each observation by
%multiplying the posterior mean of $\bm Z$ by the posterior mean of $\bm A$. We
%will call this matrix the posterior mean of $\bm {ZA}$. Each row in this matrix
%will reveal the latent structures generating the observation in the respective
%row of $\bm X$. Figure 8 shows the latent structure learned from the data. 
%For each observation, the data and latent structures are layed side by side
%for a visual comparison. The features learned were similar to the images
%created by the ten Stat666 students.
%
%
%\begin{figure}\begin{center}
%  \caption{The latent features for each student.}
%  %\includegraphics[height=1\textwidth]{../code/draw.post.out/postA66.pdf}
%  \includegraphics[height=1\textwidth]{../code/draw.post.out/postFriends.pdf}
%\end{center}\end{figure}
%
%
%\section{Conclusions}
%The IBP provides a flexible distribution on sparse binary matrices. The
%distribution can be extremely valuable when used as a prior on binary matrices
%in latent feature models. Using Gibbs sampling, the number of latent features
%can be quickly learned and the features can be discovered. Prior distributions
%can be set on $\sigma_X$ and $\sigma_A$ to improve performance. In practice,
%these parameters are unknown so they should be modeled. Areas of practical
%application for the IBP are still being studied, and requires further
%investigation.
%
%
%\newpage
%\section{Appendix -  Code \& Data}
%  \subsection{generateData.R}
%  \verbatiminput{../code/generateData.R}
%  \subsection{ibp.R}
%  \verbatiminput{../code/ibp.R}
%  \subsection{gibbs.R}
%  \verbatiminput{../code/gibbs.R}
%  \subsection{rfunctions.R}
%  \verbatiminput{../code/rfunctions.R}
%  \subsection{patterns.R}
%  \verbatiminput{../code/patterns.R}
%  \subsection{Data}
%  The data can be downloaded at: \\
%  \noindent https://github.com/luiarthur/Fall2014/blob/master/Stat666/Final/code/Y.dat\\
%  \textbf{Arthur! MAKE SURE THIS IS CURRENT!}

\end{document}
